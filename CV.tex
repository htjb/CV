\documentclass{article}
\usepackage[utf8]{inputenc}
\usepackage[margin=1.25cm]{geometry}
\usepackage{graphicx}
\usepackage{wrapfig}
\usepackage{amssymb}
\usepackage{amsmath}
\usepackage[colorlinks=True]{hyperref}
\usepackage[backend=bibtex, sorting=ydnt, maxnames=99]{biblatex}
\hypersetup{urlcolor=blue}
\usepackage{cleveref}
\usepackage[font=footnotesize]{caption}
\usepackage[para]{footmisc}
\usepackage{fontawesome} % for icons
\usepackage{titlesec} % for custom section titles

\addbibresource{ref-me.bib}

\usepackage{array, xcolor}
\definecolor{lightgray}{gray}{0.8}

\newcolumntype{L}{>{\raggedleft}p{0.14\textwidth}}
\newcolumntype{R}{p{0.8\textwidth}}

\newcolumntype{l}{>{\raggedleft}p{0.25\textwidth}}
\newcolumntype{r}{p{0.71\textwidth}}

\newcommand\VRule{\color{lightgray}\vrule width 0.5pt}

\titleformat{\section}{\large\bfseries}{\thesection}{1em}{\color{blue}}[{\titlerule[0.5pt]}]
\titleformat{\subsection}{\normalsize\bfseries}{\thesection}{2em}{\color{blue}}

\titlespacing{\section}{0pt}{6pt}{3pt} % Adjust spacing: {left}{before}{after}

\titlespacing{\subsection}{0pt}{4pt}{2pt} % Adjust spacing for subsections

\titleformat{\subsubsection}  % which section command to format
  {\fontsize{10}{12}\sffamily} % format for whole line
  {\S\thesubsection} % how to show number
  {2em} % space between number and text
  {\color{blue}} % formatting for just the text

\titlespacing{\subsubsection}{0pt}{4pt}{2pt} % Adjust spacing for subsections

\begin{document}

\small

\begin{center}
\Large
\textbf{{\color{blue}Harry T. J. Bevins, PhD}}\\
\large
Kavli Fellow, University of Cambridge
\end{center}

\begin{minipage}[ht]{0.6\linewidth}
	\faEnvelope~~\textbf{Email}: htjb2@cam.ac.uk / harrytjbevins@yahoo.co.uk\\
	\faGithub~~\textbf{Github:} \url{www.github.com/htjb}\\
	\faGlobe~~\textbf{Website:} \url{www.harrybevins.co.uk}
\end{minipage}
\begin{minipage}[ht]{0.3\linewidth}
	\begin{flushright}
	Kavli Institute for Cosmology, \\
    University of Cambridge,\\
    Madingley Road,\\
    Cambridge, CB3 0HA\\
    \textbf{Mobile Tel}: 07729396746 \\
	\end{flushright}
\end{minipage}

\section*{Summary}

\noindent Machine learning and inference expert working at the forefront of cosmological data analysis and 21-cm cosmology:
\begin{itemize}
    \item PhD begun in 2019
    \item  32 international conference talks, including 3 invited contributions, one talk at the European AI for Fundamental Physics Conference. Two additional upcoming talks.
    \item 11 first author papers and 11 contributing author papers, including two publications in Nature Astronomy
    \item Supervision of 5 PhD students, 5 Masters students and 1 summer student since October 2023
    \item Supervision of undergraduate Relativity and computing courses and lecturing for Data Intensive Science MPhil
    
\end{itemize}

\noindent \textit{machine learning, bayesian inference, simulation based inference, 21-cm cosmology, tension quantification, parameter estimation, signal emulation}

\section*{Education}
\begin{tabular}{L!{\vrule}R}
	October 2019 - September 2023 & \textbf{PhD in Physics}, Cavendish Astrophysics, University of Cambridge. \href{https://github.com/htjb/Thesis}{\textit{A Machine Learning-enhanced Toolbox for Bayesian 21-cm Data Analysis and Constraints on the Astrophysics of the Early Universe}} Supervised by Prof. Eloy de Lera Acedo, Dr Will Handley and Prof. Anastasia Fialkov. \\
	2015 - 2019 & \textbf{M.Phys. Physics with Astrophysics, $\mathbf{1^{st}}$ Class}, University of Manchester. \href{https://www.sciencedirect.com/science/article/pii/S0927650520301043?via\%3Dihub}{\textit{An updated estimate of the cosmic radio background and implications for ultra-high-energy photon propagation}}. Performed with I.C. Ni\c{t}u and supervised by Prof. Anna Scaife and Dr Justin Bray. \\
	%2008 - 2015 & Churchdown School Academy, Gloucestershire. \\ &\textbf{A-Levels:} 3 A grades, \textbf{AS-levels:} 2 A grades, \textbf{GCSE:} 7 A$^*$ grades, 3 A grades and 1 B. \\ & Highest GCSE and A-Level grades in year group.
\end{tabular}

\section*{Employment}
\begin{tabular}{L!{\vrule}R}
    January 2025 - & \textbf{Company Director}, \href{https://infersolltd.github.io}{InferSol Ltd}, Cambridge. \\
    October 2023 - & \textbf{Post-Doctoral Research Associate}, St Edmunds College, University of Cambridge. \\
    September 2023 - & \textbf{Kavli Junior Research Fellow}, Kavli Institute for Cosmology and Cavendish Astrophysics, University of Cambridge. \\
    April - September 2023 & \textbf{Post Doctoral Research Assistant}, Cavendish Astrophysics, University of Cambridge. Development of calibration techniques for 21-cm cosmology experiments. Reporting to Prof. Eloy de Lera Acedo.\\
    June-August 2018 & \textbf{Summer Research Project:} Investigating estimates of the Cosmic Radio Background. Supervised by Prof. Anna Scaife and Dr Justin Bray. University of Manchester. \\
\end{tabular}

\section*{Grants and Awards}

\begin{tabular}{l!{\vrule}r}
    DiRAC RAC17 2025 HPC Resources (3.22 Million CPU hrs $\approx$ £32,200) & Principle Investigator on ``Beam Modelling and Emulation for the Radio Experiment for the Analysis of Cosmic Hydrogen"\\
    UKRI Autumn 2024 HPC Resources (15933 Node hrs $\approx$ 2.3 Million CPU hrs on Isambard 3) & Principle Investigator on ``s(t)emu: Next generation emulators for cosmology and astrophysics" \\
    DiRAC Seedcorn (100k CPU hrs = £1000) & Constraining the Stochasticity of Star Formation with Simulation Based Inference \\
    DiRAC Seedcorn (100k CPU and 1000 GPU hrs =£1550) & Temperature dependent normalizing flows and their relationship to Nested Sampling \\
    Kavli Junior Fellowship 2023 (Accepted) & Three year independent research post with £10,000 per year research budget. \\
    Texas Center for Cosmology and Astroparticle Physics Fellowship 2023 (Declined) & Two year postdoctoral fellowship at University of Texas Austin with Prof. Julian Munoz. \\
    Trottier Space Institute Fellowship 2023 (Declined) & Two year postdoctoral fellowship at McGill University with Prof. Cynthia Chiang and Prof. John Sievers. \\
    Fitzwilliam Society Trust Research Fund 2022 & Financial support to help attend and present at the 5th Global 21-cm Workshop. \\
	Fitzwilliam College Student Opportunities Award 2022 & Financial support awarded to help attend and present at 41st MaxEnt22 Conference. \\
	Fitzwilliam College Senior Scholarship 2022 & Financial support awarded in recognition of excellent academic achievements. \\
	Fitzwilliam College Senior Scholarship 2020 & Financial support awarded in recognition of resilience, hard work and achievements in PhD studies during 2020.
\end{tabular}

\section*{Student Supervision}

\subsection*{PhD Students}

\begin{tabular}{L!{\vrule}R}
    October 2024 - & Co-supervision of Daniel Robins with Eloy de Lera Acedo and Dominic Anstey. Data analysis and foreground science for 21-cm cosmology.\\
    October 2024 - & Co-supervision of Adarsh Kumar Dash with Eloy de Lera Acedo and Dominic Anstey. Data analysis and instrumentation for 21-cm cosmology.\\
    January 2024 - & Co-supervision of Aleksandra Dragovic with Eloy de Lera Acedo and Dominic Anstey. Data analysis and foreground science for 21-cm cosmology. \\
    January 2024 - & Co-supervision of Saswata Dasgupta with Anastasia Fialkov, Dominic Anstey and Eloy de Lera Acedo. Calibration methods for global 21-cm experiments. \\
	October 2023 - & Co-supervision of Sam Leeny with Eloy de Lera Acedo and Will Handley. Machine learning calibration for 21-cm cosmology, inference techniques for fast radio bursts and supernova. \\
\end{tabular}

\subsection*{Full-Time Masters Students}

\begin{tabular}{L!{\vrule}R}
    October 2024 - & Co-Supervision of Gabriella Rajpoot with Eloy de Lera Acedo and Dominic Anstey. Impact of satellite constellations on 21-cm observations. \\
    January 2025 - & Co-Supervision of Adele Chu with Eloy de Lera Acedo and Dominic Anstey. Calibration and instrument design for 21-cm cosmology.
\end{tabular}

\subsection*{Masters Project Supervision}

\begin{tabular}{L!{\vrule}R}
    October 2024 - & Co-Supervision of Charlotte Priestley  with Will Handley. Mutual information with neural ratio estimation and high dimensional density estimation. \\
    October 2024 - & Co-supervision of Patrick Whitman with Anastasia Fialkov. Identifying ignatures of heating mechanisms on 21cm signal with machine learning. \\
    October 2024 - & Co-supervision of Lehan Li with Kaisey Mandel and Matt Grayling. Hierarchical modelling techniques for supernova studies with normalising flows. \\
    October 2023 - May 2024 & Co-Supervision of Nora Gavrea with Anastasia Fialkov. Theoretical modelling of the Dark Ages 21-cm signal. Now PhD student in School of Mathematics, University of Leeds. \\
\end{tabular}

\subsection*{Summer Students}

\begin{tabular}{L!{\vrule}R}
    July 2024 - September 2024 & Co-Supervision of Charlotte Priestley  with Will Handley. Mutual Information estimation with Neural Ratio Estimation. \\
\end{tabular}

\section*{Lecturing}

\begin{tabular}{L!{\vrule}R}
    Lent 2024 and Lent 2025 & Data Driven Radio Astronomy in the SKA era (2 Lectures), Mphil Data Intensive Science Minor Module 2024, University of Cambridge. Lecturing on \href{https://github.com/htjb/Talks/raw/master/Lectures/MPhil_Data_Intensive_Science_Lectures/Lecture-15.pdf}{Signal Emulation} and \href{https://github.com/htjb/Talks/raw/master/Lectures/MPhil_Data_Intensive_Science_Lectures/Lecture-16.pdf}{Simulation Based Inference} for Astrophysics and Cosmology.
\end{tabular}

\section*{Small Group Teaching}

\begin{tabular}{>{\raggedleft}p{0.2\textwidth}!{\vrule}p{0.76\textwidth}}
	October 2021 - April 2022 & Demonstration of Part IA Scientific Computing for 32 hours, University of Cambridge.\\
	October - December 2020 & Part II Relativity Supervision of 16 students,  University of Cambridge. \\
	July - August 2020 & Formative Marking of Part II Relativity Papers for 56 students, University of Cambridge. \\
	October - December 2019 & Part II Relativity Supervision of 6 students, University of Cambridge.
\end{tabular}

\section*{Institutional Responsibilities}

\begin{tabular}{L!{\vrule}R}
    October 2023 - & Co-organiser of Cambridge-wide bi-monthly 21-cm discussion group \\
    December 2023 - & Co-organiser of Cambridge-wide bi-monthly Astrostatistics discussion group \\
\end{tabular}

\section*{Conference Organisation}

\begin{tabular}{l!{\vrule}r}
    \href{https://eas.unige.ch/EAS2025//session.jsp?id=SS22}{\textbf{EAS Special Session, Modelling the first billion years}} June 2025 & Member of Science Organising Committee, Cork Ireland \\
    \href{https://sites.google.com/asu.edu/8g21cmworkshop}{\textbf{8th Global 21-cm Workshop}} October 2025 & Member of the Science Organising Committee, Caltech, California, United States. \\
    \href{https://www.kicc.cam.ac.uk/events/kavli-science-themed-meetings/cosmological-inference-high-dimension}{\textbf{Cosmological Inference in High Dimensions, Kavli Focus Meeting}} November 2024 & Member of Organising Committee, Univeristy of Cambridge \\
    \href{https://sites.google.com/view/global-21-cm-workshop}{\textbf{7th Global 21-cm Workshop}} October 2024 & Member of the Science Organising Committee, session and discussion chair, Raman Research Institute, Bengaluru, India. \\
    \href{https://www.kicc.cam.ac.uk/events/kavli-science-themed-meetings/science-21-cm-hydrogen-line}{\textbf{Science with the 21-cm line, Kavli Science Focus Meeting}} February 2024 & Member of Organising Committee and session chair, University of Cambridge. \\
    \href{https://www.ursi-gass2023.jp}{\textbf{URSI GASS 2023}} September 2023 & Member of Organising Committee and session co-chair for the session ``21-cm cosmology: Dark Ages, Cosmic Dawn and the Epoch of Re-ionization", Sapporo, Japan. \\
    \href{https://global21cmworkshop.org/2023-ifpu/}{\textbf{6th Global 21-cm Workshop}} September 2023 & Member of the Science Organising Committee, Trieste \\
    \href{https://nam2023.org}{\textbf{National Astronomical Meeting}} July 2023 & Member of Science Organising Committee and session chair for the session “21-cm cosmology: Current Status, Challenges and Prospects”, Cardiff University. \\
	\href{http://cavgradcon.soc.srcf.net/index.html}{\textbf{Cavendish Graduate Conference}} November 2021 & Member of Organising Committee, Cavendish Astrophysics, University of Cambridge.\\
	\href{https://sites.google.com/view/third21cmglobalworkshop/home?authuser=0}{\textbf{3\textsuperscript{rd} Global 21-cm Workshop}} October 2020 & Member of Local Organising Committee, Cavendish Astrophysics, University of Cambridge.	
\end{tabular}

\section*{Peer Review}
\noindent One paper for Physical Reviews Research, one paper for Physical Reviews D, two papers for the Journal of Open Source Software, one paper for Advances in Space Research, one for Journal of Cosmology and Astroparticle Physics, one for The Astronomical Journal, one for the Open Journal of Astrophysics and one for Monthly Notices of the Royal Astronomical Society (see \href{https://www.webofscience.com/wos/author/record/3920858}{Web of Science}).

\section*{Computational Skills}

\begin{tabular}{L!{\vrule}R}
	Programming and & Experienced: Python, Tensorflow, Pytorch \\
    Software & Intermediate: JAX \\
    OS & Experienced: Linux, HPC, MacOS \\
	Markup & Experienced: LATEX , Markdown\\
	Languages & Intermediate: HTML, CSS, reStructuredText
\end{tabular}

\section*{Software}

\begin{tabular}{L!{\vrule}R}
	\href{https://github.com/htjb/margarine}{margarine} & Main author and maintainer: Posterior Sampling and Marginal Bayesian Statistics using Normalizing Flows. \\
	\href{https://github.com/htjb/globalemu}{globalemu} & Main author and maintainer: Robust and fast emulation of the Global 21-cm signal using Tensorflow. \\
	\href{https://github.com/htjb/maxsmooth}{maxsmooth} & Main author and maintainer: Fast Derivative Constrained Function fitting.\\
     \href{https://github.com/htjb/beta-flows}{betaflows} & Main author and maintainer: Temperature dependent Normalizing Flows using Tensorflow. \\
	\href{https://github.com/williamjameshandley/anesthetic}{anesthetic} & Contributor: Added feature to plot shaded confidence regions under 1D posterior probability plots.\\
	\href{https://github.com/htjb/arXivSearcher}{arXivSearcher} & Main author and maintainer: Terminal based arXiv search tool.
\end{tabular}

\section*{In the Media}

\begin{tabular}{L!{\vrule}R}
    August 2024 & \href{https://www.kicc.cam.ac.uk/aboutus/kicc-annual-reports}{Getting The Most Out Of Our Inference Products With Machine Learning}. Contribution to the Kavli Institute for Cosmology, Cambridge 2023 annual report. Co-authored with Will Handley. \\
	November 2022 & \href{https://www.cam.ac.uk/research/news/non-detection-of-key-signal-allows-astronomers-to-determine-what-the-first-galaxies-were-and-werent}{Non-detection of key signal allows astronomers to determine what the first galaxies were – and weren’t – like}. Press release for SARAS3 astrophysical constraints from the University of Cambridge. See also an article in \href{https://www.independent.co.uk/space/galaxies-beginning-cosmic-dawn-signal-b2234570.html?amp}{The Independent} and \href{https://nature.altmetric.com/details/139089769/news}{Altmetric} for a summary of other articles. \\
	July 2022 & \href{https://www.cam.ac.uk/research/news/astronomers-develop-novel-way-to-see-the-first-stars-through-the-fog-of-the-early-universe}{Astronomers develop novel way to ‘see’ first stars through fog of early Universe}. Press release for the REACH mission paper led by Eloy de Lera Acedo. See \href{https://nature.altmetric.com/details/132932581/news}{Altmetrics} for a summary of other articles.
\end{tabular}

\pagebreak

\section*{Talks}

\subsection*{Invited Talks and Panel Discussions}
\begin{tabular}{L!{\vrule}R}
September 2024 & \href{https://github.com/htjb/Talks/raw/master/Talks/St-Andrews-Seminar-Sept-2024/standrews24.pdf}{\textit{Marginal Bayesian Inference with Normalizing Flows}}, Lunch Time Seminar, University of St Andrews, Scotland. \\ 
August 2024 & \href{https://indico.cern.ch/event/1293041/contributions/5970077/}{\textit{Calibrating Tension Statistics with Neural Ratio Estimation}}, XVIth Quark Confinement and the Hadron Spectrum, University of Adelaide, Cairns, Australia. \\
December 2023 & \href{https://github.com/htjb/Talks/raw/master/Talks/cambridge-lmu/Cambridge-LMU.pdf}{\textit{Forward Modelling in 21-cm cosmology}}, \href{https://indico.physik.uni-muenchen.de/event/394/}{Cambridge - LMU Forward Modelling Cosmology Meeting}, Max-Planck-Institute for Extraterrestrial Physics, Munich, Germany.\\
December 2023 & Panellist on `Forward simulating the Universe - Opportunities and Challenges?' panel, \href{https://indico.physik.uni-muenchen.de/event/394/}{Cambridge - LMU Forward Modelling Cosmology Meeting}, Max-Planck-Institute for Extraterrestrial Physics, Munich, Germany. \\
September 2023 & Panellist on `Instrumentation \& Methods' panel, \href{https://global21cmworkshop.org/2023-ifpu/}{6$^{th}$ Global 21-cm Workshop}, Fundamental Physics of the Universe (IFPU), Trieste, Italy \\
\end{tabular}

\subsection*{Conference, Seminar and Workshop Talks}
\begin{tabular}{L!{\vrule}R}
    June 2025 & \textit{\href{}{On the accuracy of posterior recovery with neural network emulators}}, \href{https://agenda.infn.it/event/43565/}{European AI for Fundamental Physics Conference, Sardinia} \\
    May 2025 & \textit{\href{}{Constraining the stochasticity of star formation across galaxy populations with simulation based inference}, Simulation Based Inference for Galaxy Evolution}, University of Bristol, UK \\
    November 2024 & \textit{\href{https://github.com/htjb/Talks/raw/master/Talks/Kavli-high-dimensional-analysis-20NOV24/posterior-validation.pdf}{On the accuracy of posterior recovery with neural network emulators}}, Cosmological inference in high dimensions, Kavli Science Focus Meeting, University of Cambridge, UK \\
    November 2024 & \textit{\href{https://github.com/htjb/Talks/raw/master/Talks/CosmoCube-JPL-Workshop-2024/calibration-jpl-cosmocube.pdf}{Calibration for 21-cm cosmology}}, CosmoCube@JPL workshop, Jet Propolsion Laboratory, California, US \\
    October 2024 & \textit{\href{https://github.com/htjb/Talks/raw/master/Talks/21cm-Meeting-14OCT24/posterior-validation.pdf}{On the accuracy of posterior recovery with neural network emulators}}, 21 cm Discussion Group, University of Cambridge, Cambridge, UK \\
    October 2024 & \textit{REACH Calibration Overview}, REACH Annual General Meeting, Mahabaleshwar, India \\
    October 2024 & \textit{Beam Modelling and Emulation for REACH}, REACH Annual General Meeting, Mahabaleshwar, India \\
    October 2024 & \textit{\href{https://github.com/htjb/Talks/raw/master/Talks/7th-Global-Workshop/margarine.pdf}{Marginal Bayesian Statistics and Post Processing of Bayesian Constraints}}, \href{https://sites.google.com/view/global-21-cm-workshop}{7th Global 21-cm Workshop}, Raman Research Institute, Bengaluru, India \\
    August 2024 & \textit{\href{https://github.com/htjb/Talks/tree/master/Talks/brisbane_davis_group_meeting_2024}{Machine Learning Enhanced Bayesian Inference for Cosmology}}, Seminar Talk, University of Queensland, Brisbane, Australia \\
    July 2024 & \textit{\href{https://github.com/htjb/Talks/raw/master/Talks/ML4ASTRO-Tensionnets/tensions.pdf}{Calibrating Tension Statistics with Neural Ratio Estimation}}, \href{https://indico.ict.inaf.it/event/2690/}{ML4ASTRO2}, Catania, Sicily \\
    May 2024 & \textit{\href{https://github.com/htjb/Talks/raw/master/Talks/Amsterdam-EuCAIF-2024/tensions.pdf}{Calibrating Tension Statistics with Neural Ratio Estimation}}, \href{https://indico.nikhef.nl/event/4875/}{European AI for Fundamental Physics Conference}, Amsterdam \\
    February 2024 & \textit{\href{https://github.com/htjb/Talks/raw/master/Talks/RAS_Multi-tracer_View/21cmCosmo.pdf}{21-cm cosmology as probe of the high redshift Universe}}, \href{https://ras.ac.uk/events-and-meetings/ras-meetings/multi-tracer-view-galaxies-first-few-billion-years}{A Multi-tracer view of galaxies in the first few billion years}, Royal Astronomical Society, London \\
    November 2023 & \textit{\href{https://github.com/htjb/Talks/raw/master/Talks/St_Edmunds_Research_Salon_Nov_2023/research_salon.pdf}{Observing the first stars and galaxies}}, Research Salon, St Edmunds College, Cambridge \\
    October 2023 & \textit{\href{https://github.com/htjb/Talks/raw/master/Talks/KICC_Intro_23/kicc_intro_23.pdf}{Novel Statistical Methods and 21-cm cosmology}}, Intro to KICC, University of Cambridge. \\
    September 2023 & \textit{REACH Calibration Software Update}, REACH Annual Meeting, University of Malta, Malta \\
    September 2023 & \textit{\href{https://github.com/htjb/Talks/tree/master/Talks/REACH_Malta/joint_analysis_margarine.pdf}{Joint analysis constraints from 21-cm probes with normalizing flows}}, REACH Annual Meeting, University of Malta, Malta \\
    August 2023 & \href{https://github.com/htjb/Talks/raw/master/Talks/URSI_2023/joint_analysis_margarine.pdf}{\textit{Joint analysis constraints on the physics of the first galaxies from upper limits on the 21-cm power spectrum and sky-averaged signal}}, 21-cm cosmology: Dark Ages, Cosmic Dawn and the Epoch of Re-ionization, URSI General Assembly and Scientific Symposium, Sapporo, Japan. \\
    August 2023 & \textit{Constraints on Superconducting Cosmic Strings from 21-cm cosmology} on behalf of Thomas Gessy-Jones, 21-cm cosmology: Dark Ages, Cosmic Dawn and the Epoch of Re-ionization, URSI General Assembly and Scientific Symposium, Sapporo, Japan. \\
    May 2023 & \href{https://github.com/htjb/Talks/blob/master/Talks/KICC_EoR/Bevins_KICC_EoR.pdf}{\textit{Utilizing Normalizing Flows to enhance our Bayesian workflows}}, Astrostatistics and Astro-ML, KICC, University of Cambridge. \\
    May 2023 & \href{https://github.com/htjb/Talks/blob/master/Talks/KICC_EoR/Bevins_KICC_EoR.pdf}{\textit{Joint analysis constraints on the physics of the first galaxies with low frequency radio astronomy data}}, A Multi-scale View of the EoR, KICC, University of Cambridge. \\
    February 2023 & \href{https://github.com/htjb/Talks/blob/master/Talks/DARA_REACH_Workshop/DARA_Workshop.pdf}{\textit{Machine Learning for 21-cm cosmology}}, Development in Africa with Radio Astronomy REACH Workshop, University of Stellenbosch. \\
    October 2022 & \href{https://global21cmworkshop.org/2022-berkeley/}{\textit{Astrophysics from the SARAS3 non-detection of the global 21-cm signal}}, 5th Global 21-cm Workshop, University of California Berkeley. \\
	July 2022 & \href{https://github.com/htjb/Talks/blob/master/Talks/MaxEnt22_margarine_June_2022/MaxEnt_slides.pdf}{\textit{Marginal Bayesian Statistics with Masked Autoregressive Flows and Kernel Density Estimators}}, 41st MaxEnt22 Conference, Institut Henri Poincar\'e, Paris. \\
	June 2022 & \href{https://github.com/htjb/Talks/blob/master/Talks/ESA_globalemu_July_2022/AUDITORIUM\%203A_SS23a_0940_Bevins.pdf}{\textit{globalemu: Novel and robust emulation of 21-cm signals from the Epoch of Reionization}}, Special Session - Towards the SKA Observatory:Artificial Intelligence in Radio Astronomy, European Astronomical Society Annual Meeting, Valencia. \\
\end{tabular}
\subsection*{Conference, Seminar and Workshop Talks}
\begin{tabular}{L!{\vrule}R}
	April 2022 & \href{https://www.kicc.cam.ac.uk/events/kavli-science-themed-meetings/observational-and-theoretical-21-cm-cosmology}{\textit{Constraining the Astrophysics of the Early Universe using the SARAS Instrumentation}}, Observational and Theoretical 21-cm cosmology, Kavli Meeting, University of Cambridge \\
	March 2022 & \textit{A comprehensive Bayesian re-analysis of the SARAS2 data from the Epoch of Reionization}, SAZERAC 21-cm Gulp (Online) \\
	January 2022 & \href{https://www.youtube.com/watch?v=BFwia93NuAc&list=PLp95u5tgS_YUkFaLATBQpLajJzO5ljN5u&index=5}{\textit{globalemu: A novel and robust approach to emulating the global 21-cm signal with neural networks}}, SAZERAC-SIP:Learning the high-redshift universe (Online) \\
	December 2021 & \href{https://www.youtube.com/watch?v=93KCp7rHcGA&list=PLZL7YmXBBHPDCyNfJcWwP78GgacY_Og4E&index=23}{\textit{A Bayesian re-analysis of the sky-averaged 21-cm experimental data from SARAS2}}, Lightening talk, Science at Low Frequencies VIII (Online), University of Amsterdam \\
	October 2021 & \href{https://www.youtube.com/watch?v=862NuVyF33k&list=PLF7c7ri2hrnGlwbn4JLc0PWbncSeARdTP&index=4}{\textit{globalemu: novel and robust global 21-cm signal emulation}}, 4th Global 21-cm Workshop (Online), University of Colorado \\
	December 2020 & \href{https://www.youtube.com/watch?v=Yw6_IInwTNE}{\textit{maxsmooth and its applications in science at low frequencies}}, Lightening talk, Science at Low Frequencies VII (Online), University of Amsterdam \\
	October 2020 & \textit{maxsmooth and its applications to 21-cm cosmology}, 3rd Global 21-cm Workshop (Online), University of Cambridge \\
\end{tabular}


\section*{Posters Presentations}

\begin{tabular}{L!{\vrule}R}
	March 2021 & \href{https://github.com/htjb/Talks/blob/master/Posters/SKA_globalemu_March_2021/globalemu.pdf}{\textit{globalemu: A novel and robust approach for emulating the sky averaged 21-cm signal from the cosmic dawn and epoch of reionization}}, A Precursor View of the SKA Sky, Virtual Conference\\
	December 2019 & \href{https://drive.google.com/file/d/1dvgumyu4cXxXqoYxikU3DKOa4u_gpGzn/view}{\textit{REACH: Radio Experiment for the Analysis of Cosmic Hydrogen}}, Science At Low Frequencies VI, Arizona State University \\
	November 2019 & \href{https://github.com/htjb/Talks/blob/master/Posters/Cav_Graduate_Conf_REACH_Nov_2019/REACH_poster.pdf}{\textit{REACH: Radio Experiment for the Analysis of Cosmic Hydrogen}}, Cavendish Graduate Conference, University of Cambridge
\end{tabular}

\section*{Publications}
\noindent \textbf{Citations: 364, h-index: 11, as of April 2025}

\noindent \href{http://arxiv.org/a/bevins_h_1}{arxiv.org/a/bevins\_h\_1}

\noindent \href{https://ui.adsabs.harvard.edu/search/p_=0&q=author\%3A\%22Bevins\%2C\%20H.\%20T.\%20J.\%22&sort=date\%20desc\%2C\%20bibcode\%20desc}{ADS Abstracts}

\noindent \href{https://scholar.google.com/citations?user=03_3YwQAAAAJ&hl=en}{Google Scholar}

\defbibheading{bibintoc}[]{%
  % No title, no rule
}

\nocite{*}
\printbibliography[title={},heading=bibintoc]


\end{document}
