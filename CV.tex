\documentclass{article}
\usepackage[utf8]{inputenc}
\usepackage[margin=1.5cm]{geometry}
\usepackage{graphicx}
\usepackage{wrapfig}
\usepackage{natbib}
\usepackage{amssymb}
\usepackage{amsmath}
\usepackage[colorlinks=True]{hyperref}
\hypersetup{urlcolor=blue}
\usepackage{cleveref}
\usepackage[font=footnotesize]{caption}
\usepackage[para]{footmisc}

\usepackage{array, xcolor}
\definecolor{lightgray}{gray}{0.8}

\newcolumntype{L}{>{\raggedleft}p{0.14\textwidth}}
\newcolumntype{R}{p{0.8\textwidth}}

\newcolumntype{l}{>{\raggedleft}p{0.25\textwidth}}
\newcolumntype{r}{p{0.71\textwidth}}

\newcommand\VRule{\color{lightgray}\vrule width 0.5pt}

\crefformat{figure}{Fig.~#2#1#3}
\crefformat{equation}{equation~#2#1#3}
%\crefformat{section}{section~#2#1#3}
\crefformat{table}{Tab.~#2#1#3}
\crefformat{appendix}{appendix~#2#1#3}
\crefmultiformat{figure}{Figs.~#2#1#3}{~and~#2#1#3}%
    {,~#2#1#3}{,~#2#1#3}
\crefrangeformat{figure}{Figs.~(#3#1#4--#5#2#6)}

\begin{document}

\small

\begin{center}
\huge
\textbf{Harry T. J. Bevins, M.Phys.}
\end{center}

\begin{minipage}[ht]{0.6\linewidth}
	\textbf{Primary Email}: htjb2@cam.ac.uk \\
	\textbf{Secondary Email}: harrytjbevins@yahoo.co.uk \\
	\textbf{Github:} \url{www.github.com/htjb}\\
	\textbf{Website:} \url{harrybevins.co.uk}
\end{minipage}
\begin{minipage}[ht]{0.3\linewidth}
	\begin{flushright}
	Fitzwilliam College, Storey's Way,\\
	Cambridge, Cambridgeshire,\\
	England,\\
	CB3 0DG \\
    \textbf{Mobile Tel}: 07729396746 \\
	\end{flushright}
\end{minipage}

\section*{Employment}
\begin{tabular}{L!{\vrule}R}
    September 2023 - September 2026 & \textbf{Kavli Junior Research Fellowship}, Kavli Institute for Cosmology and Cavendish Astrophysics, University of Cambridge. \\
    April 2023 - September 2023 & \textbf{Post Doctoral Research Assistant}, Cavendish Astrophysics, University of Cambridge. Development of calibration techniques for 21-cm Cosmology experiments. Supervised by Dr Eloy de Lera Acedo.\\
    June-August 2018 & \textbf{Summer Research Project:} Investigating estimates of the Cosmic Radio Background. Supervised by Prof. Anna Scaife and Dr Justin Bray. University of Manchester. \\
\end{tabular}

\section*{Education}
\begin{tabular}{L!{\vrule}R}
	October 2019 - September 2023 & \textbf{PhD in Physics}, Cavendish Astrophysics, University of Cambridge. Development and application of novel data analysis techniques for 21-cm Cosmology experiments. Supervised by Dr Eloy de Lera Acedo, Dr Will Handley and Dr Anastasia Fialkov. \\
	2015 - 2019 & \textbf{M.Phys. Physics with Astrophysics, $\mathbf{1^{st}}$ Class}, University of Manchester. Dissertation involved producing an updated estimate of the Cosmic Radio Background from observations and theory. Performed with I.C. Ni\c{t}u and supervised by Prof. Anna Scaife and Dr Justin Bray. \\
	2008 - 2015 & Churchdown School Academy, Gloucestershire. \\ &\textbf{A-Levels:} 3 A grades, \textbf{AS-levels:} 2 A grades, \textbf{GCSE:} 7 A$^*$ grades, 3 A grades and 1 B. \\ & Highest GCSE and A-Level grades in year group.
\end{tabular}

\noindent{\Large\textbf{Publications}} \hfill \textbf{Citations: 120, h-index: 7, as of July 2023}

\hfill \href{http://arxiv.org/a/bevins_h_1}{arxiv.org/a/bevins\_h\_1}

\hfill \href{https://ui.adsabs.harvard.edu/search/p_=0&q=author\%3A\%22Bevins\%2C\%20H.\%20T.\%20J.\%22&sort=date\%20desc\%2C\%20bibcode\%20desc}{ADS Abstracts}

\subsection*{First Author}

\begin{tabular}{L!{\vrule}R}
    May 2023 & \textbf{H. T. J. Bevins}, W. J. Handley. \href{https://arxiv.org/abs/2305.02930}{\textit{Piecewise Normalizing Flows}}, arXiv:2305.02930. \\
    January 2023 & \textbf{H. T. J. Bevins}, S. Heimersheim, I. Abril-Cabezas, A. Fialkov, E. de Lera Acedo, W. J. Handley, S. Singh, R. Barkana. \href{https://arxiv.org/abs/2301.03298}{\textit{Joint analysis constraints on the physics of the first galaxies with low frequency radio astronomy data.}}, arXiv:2301.03298. \\
    November 2022 & \textbf{H. T. J. Bevins}, A. Fialkov, E. de Lera Acedo, W. J. Handley, S. Singh, R. Subrahmanyan, R. Barkana. \href{https://www.nature.com/articles/s41550-022-01825-6}{\textit{Astrophysical Constraints from the SARAS3 non-detection of the Cosmic Dawn Sky-Averaged 21-cm Signal.}} \textbf{Nature Astronomy}, https://doi.org/10.1038/s41550-022-01825-6.\\
	October 2022 & \textbf{H. T. J. Bevins}, W. J. Handley, P. Lemos, P. Sims, E. de Lera Acedo, A. Failkov. \href{https://www.mdpi.com/2673-9984/5/1/1#abstract}{\textit{Marginal Bayesian Statistics Using Masked Autoregressive Flows and Kernel Density Estimators with Examples in Cosmology.}} Physical Sciences Forum, Volume 5, Issue 1, proceedings to MaxEnt22, arXiv:2207.11457. \\
	May 2022 & \textbf{H. T. J. Bevins}, W. J. Handley, P. Lemos, P. Sims, E. de Lera Acedo, A. Fialkov, J. Alsing. \href{https://arxiv.org/abs/2205.12841}{\textit{Removing the fat from your posterior samples with \textsc{margarine}}}, arXiv:2205.12841. \\
	April 2022 & \textbf{H. T. J. Bevins}, E. de Lera Acedo, A. Fialkov, W. J. Handley, S. Singh, R. Subrahmanyan, R. Barkana. \href{https://doi.org/10.1093/mnras/stac1158}{\textit{A comprehensive Bayesian re-analysis of the SARAS2 data from the Epoch of Reionization}}, Monthly Notices of the Royal Astronomical Society, Volume 508, Issue 2, Pages 2923-2936, arXiv:2201.11531\\
	September 2021 & \textbf{H. T. J. Bevins}, W. J. Handley, A. Fialkov, E. de Lera Acedo and K. Javid. \href{https://doi.org/10.1093/mnras/stab2737}{\textit{\textsc{globalemu}: A novel and robust approach for emulating the sky-averaged 21-cm signal from the cosmic dawn and epoch of reionisation}}, Monthly Notices of the Royal Astronomical Society, Volume 513, Issue 3, Pages 4507-4526, arXiv:2104.04336 \\
	April 2021 & \textbf{H. T. J. Bevins}, W. J. Handley, A. Fialkov, E. de Lera Acedo, L. J. Greenhill, D. C. Price, \href{https://academic.oup.com/mnras/article/502/3/4405/6105349?guestAccessKey=769d7461-cc0b-4d83-9cdf-bc351056d911}{\textit{\textsc{maxsmooth}: rapid maximally smooth function fitting with applications it Global 21-cm cosmology}}, Monthly Notices of the Royal Astronomical Society, Volume 502, Issue 3, Pages 4405–4425, 
    arXiv:2007.14970\\
	November 2020 & I.C. Ni\c{t}u, \textbf{H.T.J. Bevins}, J.D. Bray, A.M.M. Scaife, \href{https://www.sciencedirect.com/science/article/pii/S0927650520301043?via\%3Dihub}{\textit{An updated estimate of the cosmic radio background and implications for ultra-high-energy photon propagation}}, Astroparticle Physics, 126, 102532, arXiv:2004.13596, \textit{Shared First Authorship}. \\
	October 2020 & \textbf{H. T. J. Bevins}, \href{https://joss.theoj.org/papers/10.21105/joss.02596}{\textit{maxsmooth: Derivative Constrained Function Fitting.}} Journal of Open Source Software, 5(54), 2596 \\
\end{tabular}

\subsection*{Contributing Author}

\begin{tabular}{L!{\vrule}R}
    July 2023 & The REACH Collaboration. \href{https://arxiv.org/abs/2307.00099}{\textit{Receiver design for the REACH global 21-cm signal experiment}}, arXiv:2307.00099. \\
	July 2022 & The REACH Collaboration. \href{https://www.nature.com/articles/s41550-022-01709-9?utm_campaign=related_content&utm_source=ASTRO&utm_medium=communities}{\textit{The REACH radiometer for detecting the 21-cm hydrogen signal from redshift $z \approx 7.5 -28$.}} \textbf{Nature Astronomy}, 6, 984–998 (2022). https://doi.org/10.1038/s41550-022-01709-9. \textit{Section lead on Detection of Unknown Systematics}. \\
	February 2022 & The REACH Collaboration. \href{https://ui.adsabs.harvard.edu/abs/2021arXiv210910098C/abstract}{\textit{Radio antenna design for sky-averaged 21 cm cosmology experiments: the REACH case}}, Journal of Astronomical Instrumentation, Volume 11, Issue 1, arXiv:2109.10098. \\
\end{tabular}

\section*{Talks}
\subsection*{Conference and Workshop Talks}
\begin{tabular}{L!{\vrule}R}
    August 2023 & \href{https://github.com/htjb/Talks/raw/master/Talks/URSI_2023/joint_analysis_margarine.pdf}{\textit{Joint analysis constraints on the physics of the first galaxies from upper limits on the 21-cm power spectrum and sky-averaged signal}}, 21-cm Cosmology: Dark Ages, Cosmic Dawn and the Epoch of Re-ionization, URSI General Assembly and Scientific Symposium, Sapporo, Japan. \\
    & \textit{Constraints on Superconducting Cosmic Strings from 21-cm Cosmology} on behalf of Thomas Gessy-Jones, 21-cm Cosmology: Dark Ages, Cosmic Dawn and the Epoch of Re-ionization, URSI General Assembly and Scientific Symposium, Sapporo, Japan. \\
    May 2023 & \href{https://github.com/htjb/Talks/blob/master/Talks/KICC_EoR/Bevins_KICC_EoR.pdf}{\textit{Utilizing Normalizing Flows to enhance our Bayesian workflows}}, Astrostatistics and Astro-ML, KICC, University of Cambridge. \\
    May 2023 & \href{https://github.com/htjb/Talks/blob/master/Talks/KICC_EoR/Bevins_KICC_EoR.pdf}{\textit{Joint analysis constraints on the physics of the first galaxies with low frequency radio astronomy data}}, A Multi-scale View of the EoR, KICC, University of Cambridge. \\
    February 2023 & \href{https://github.com/htjb/Talks/blob/master/Talks/DARA_REACH_Workshop/DARA_Workshop.pdf}{\textit{Machine Learning for 21-cm Cosmology}}, Development in Africa with Radio Astronomy REACH Workshop, University of Stellenbosch. \\
    October 2022 & \href{https://global21cmworkshop.org/2022-berkeley/}{\textit{Astrophysics from the SARAS3 non-detection of the global 21-cm signal}}, 5th Global 21-cm Workshop, University of California Berkeley. \\
	July 2022 & \href{https://github.com/htjb/Talks/blob/master/Talks/MaxEnt22_margarine_June_2022/MaxEnt_slides.pdf}{\textit{Marginal Bayesian Statistics with Masked Autoregressive Flows and Kernel Density Estimators}}, 41st MaxEnt22 Conference, Institut Henri Poincar\'e, Paris. \\
	June 2022 & \href{https://github.com/htjb/Talks/blob/master/Talks/ESA_globalemu_July_2022/AUDITORIUM\%203A_SS23a_0940_Bevins.pdf}{\textit{globalemu: Novel and robust emulation of 21-cm signals from the Epoch of Reionization}}, Special Session - Towards the SKA Observatory:Artificial Intelligence in Radio Astronomy, \textbf{European Astronomical Society Annual Meeting}, Valencia. \\
	April 2022 & \href{https://www.kicc.cam.ac.uk/events/kavli-science-themed-meetings/observational-and-theoretical-21-cm-cosmology}{\textit{Constraining the Astrophysics of the Early Universe using the SARAS Instrumentation}}, Observational and Theoretical 21-cm Cosmology, Kavli Meeting, University of Cambridge \\
	March 2022 & \textit{A comprehensive Bayesian re-analysis of the SARAS2 data from the Epoch of Reionization}, SAZERAC 21-cm Gulp (Online) \\
	January 2022 & \href{https://www.youtube.com/watch?v=BFwia93NuAc&list=PLp95u5tgS_YUkFaLATBQpLajJzO5ljN5u&index=5}{\textit{globalemu: A novel and robust approach to emulating the global 21-cm signal with neural networks}}, SAZERAC-SIP:Learning the high-redshift universe (Online) \\
	December 2021 & \href{https://www.youtube.com/watch?v=93KCp7rHcGA&list=PLZL7YmXBBHPDCyNfJcWwP78GgacY_Og4E&index=23}{\textit{A Bayesian re-analysis of the sky-averaged 21-cm experimental data from SARAS2}}, , Lightening talk, Science at Low Frequencies VIII (Online), University of Amsterdam \\
	October 2021 & \href{https://www.youtube.com/watch?v=862NuVyF33k&list=PLF7c7ri2hrnGlwbn4JLc0PWbncSeARdTP&index=4}{\textit{globalemu: novel and robust global 21-cm signal emulation}}, 4th Global 21-cm Workshop (Online), University of Colorado \\
	December 2020 & \href{https://www.youtube.com/watch?v=Yw6_IInwTNE}{\textit{maxsmooth and its applications in science at low frequencies}}, Lightening talk, Science at Low Frequencies VII (Online), University of Amsterdam \\
	October 2020 & \textit{maxsmooth and its applications to 21-cm cosmology}, 3rd Global 21-cm Workshop (Online), University of Cambridge \\
\end{tabular}

\subsection*{Internal Talks}
\begin{tabular}{L!{\vrule}R}
	October 2021 & \textit{An extensive Bayesian re-analysis of the SARAS2 data from the Epoch of Reionization}, 21-cm Group Meeting (Online), Cavendish Astrophysics, University of Cambridge \\
	February 2021 & \textit{maxsmooth and Maximally Smooth Functions}, Internal Coffee Talk (Online), Cavendish Astrophysics, University of Cambridge \\
	February 2021 & \textit{GlobalEmu: A novel and robust approach to emulating the Global 21-cm signal and applications to data from SARAS2}, 21-cm Group Meeting (Online), Cavendish Astrophysics, University of Cambridge \\
	July 2020 & \textit{maxsmooth: rapid maximally smooth function fitting with applications to 21-cm cosmology}, 21-cm Group Meeting, Cavendish Astrophysics, University of Cambridge \\
	May 2020 & \textit{An updated estimate of the cosmic radio background and implications for ultra-high-energy photon propagation}, Internal Coffee Talk, Cavendish Astrophysics, University of Cambridge \\
	May 2020 & \textit{An updated estimate of the cosmic radio background and implications for 21-cm Cosmology}, 21-cm Group Meeting, Cavendish Astrophysics, University of Cambridge
\end{tabular}

\section*{Grants and Awards}

\begin{tabular}{l!{\vrule}r}
    DiRAC Seedcorn (100k CPU and 1000 GPU hrs =£1550) & Seedcorn grant to investigate temperature dependent normalizing flows and their relationship to Nested Sampling \\
    Fitzwilliam Society Trust Research Fund 2022 & Financial support to help attend and present at the 5th Global 21-cm Workshop. \\
	Fitzwilliam College Student Opportunities Award 2022 & Financial support awarded to help attend and present at 41st MaxEnt22 Conference. \\
	Fitzwilliam College Senior Scholarship 2022 & Financial support awarded in recognition of excellent academic achievements. \\
	Fitzwilliam College Senior Scholarship 2020 & Financial support awarded in recognition of resilience, hard work and achievements in PhD studies during 2020.
\end{tabular}

\section*{Posters Presentations}

\begin{tabular}{L!{\vrule}R}
	March 2021 & \href{https://github.com/htjb/Talks/blob/master/Posters/SKA_globalemu_March_2021/globalemu.pdf}{\textit{globalemu: A novel and robust approach for emulating the sky averaged 21-cm signal from the cosmic dawn and epoch of reionization}}, A Precursor View of the SKA Sky, Virtual Conference\\
	December 2019 & \href{https://drive.google.com/file/d/1dvgumyu4cXxXqoYxikU3DKOa4u_gpGzn/view}{\textit{REACH: Radio Experiment for the Analysis of Cosmic Hydrogen}}, Science At Low Frequencies VI, Arizona State University \\
	November 2019 & \href{https://github.com/htjb/Talks/blob/master/Posters/Cav_Graduate_Conf_REACH_Nov_2019/REACH_poster.pdf}{\textit{REACH: Radio Experiment for the Analysis of Cosmic Hydrogen}}, Cavendish Graduate Conference, University of Cambridge
\end{tabular}

\section*{Peer Review}
One paper for the Journal of Open Source Software, one paper for Advances in Space Research and one for Journal of Cosmology and Astroparticle Physics (see \href{https://www.webofscience.com/wos/author/record/3920858}{Web of Science}).

\section*{In the Media}

\begin{tabular}{L!{\vrule}R}
	November 2022 & \href{https://www.cam.ac.uk/research/news/non-detection-of-key-signal-allows-astronomers-to-determine-what-the-first-galaxies-were-and-werent}{Non-detection of key signal allows astronomers to determine what the first galaxies were – and weren’t – like}. Press release for SARAS3 astrophysical constraints from the University of Cambridge. See also an article in \href{https://www.independent.co.uk/space/galaxies-beginning-cosmic-dawn-signal-b2234570.html?amp}{The Independent} and \href{https://nature.altmetric.com/details/139089769/news}{Altmetric} for a summary of other articles. \\
	July 2022 & \href{https://www.cam.ac.uk/research/news/astronomers-develop-novel-way-to-see-the-first-stars-through-the-fog-of-the-early-universe}{Astronomers develop novel way to ‘see’ first stars through fog of early Universe}. Press release for the REACH mission paper lead by Eloy de Lera Acedo. See \href{https://nature.altmetric.com/details/132932581/news}{Altmetrics} for a summary of other articles.
\end{tabular}

\section*{Teaching}

\begin{tabular}{>{\raggedleft}p{0.2\textwidth}!{\vrule}p{0.76\textwidth}}
	October 2021 - April 2022 & \textbf{Demonstration of Part IA Scientific Computing} for 32 hours, University of Cambridge.\\
	October - December 2020 & \textbf{Part II Relativity Supervision} of 16 students,  University of Cambridge. \\
	July - August 2020 & \textbf{Formative Marking of Part II Relativity Papers} for 56 students, University of Cambridge. \\
	October - December 2019 & \textbf{Part II Relativity Supervision} of 6 students, University of Cambridge.
\end{tabular}

\section*{Computational Skills}

\begin{tabular}{L!{\vrule}R}
	Programming & Experienced: Python \\
	Markup & Experienced: LATEX \\
	Languages & Intermediate: HTML, CSS, reStructuredText, Markdown
\end{tabular}

\section*{Software}

\begin{tabular}{L!{\vrule}R}
	\href{https://github.com/htjb/margarine}{margarine} & Main author and maintainer: Posterior Sampling and Marginal Bayesian Statistics. \\
	\href{https://github.com/htjb/globalemu}{globalemu} & Main author and maintainer: Robust and fast emulation of the Global 21-cm signal.\\
	\href{https://github.com/htjb/maxsmooth}{maxsmooth} & Main author and maintainer: Fast Derivative Constrained Function fitting.\\
	\href{https://github.com/williamjameshandley/anesthetic}{anesthetic} & Contributor: Added feature to plot shaded confidence regions under 1D posterior probability plots.\\
	\href{https://github.com/htjb/arXivSearcher}{arXivSearcher} & Main author and maintainer: Terminal based arXiv search tool.
\end{tabular}

\section*{Conference Organisation}

\begin{tabular}{l!{\vrule}r}
    \href{https://www.ursi-gass2023.jp}{\textbf{URSI GASS 2023}} September 2023 & Member of Organising Committee and session co-chair for the session "21-cm Cosmology: Dark Ages, Cosmic Dawn and the Epoch of Re-ionization", Sapporo, Japan. \\
    \href{}{\textbf{6th Global 21-cm Workshop}} September 2023 & Member of the Science Organising Committee, Trieste \\
    \href{https://nam2023.org}{\textbf{National Astronomical Meeting}} July 2023 & Member of Science Organising Committee and session chair for the session “21-cm Cosmology: Current Status, Challenges and Prospects”, Cardiff University. \\
	\href{http://cavgradcon.soc.srcf.net/index.html}{\textbf{Cavendish Graduate Conference}} November 2021 & Member of Organising Committee, Cavendish Astrophysics, University of Cambridge.\\
	\href{https://sites.google.com/view/third21cmglobalworkshop/home?authuser=0}{\textbf{3\textsuperscript{rd} Global 21-cm Workshop}} October 2020 & Member of Local Organising Committee, Cavendish Astrophysics, University of Cambridge.	
\end{tabular}
%\section*{References}

%\textbf{Dr Eloy de Lera Acedo}, PhD Supervisor, Head of Cavendish Radio Cosmology and Principle Investigator, Cavendish Astrophysics, University of Cambridge. \textbf{Email:}ed330@cam.ac.uk\\
%\\
%\textbf{Dr Will Handley}, PhD Supervisor, Royal Society University Research Fellow, Cavendish Astrophysics, University of Cambridge. \textbf{Email:} wh260@cam.ac.uk\\
%\\
%\textbf{Dr Anastasia Fialkov}, PhD Supervisor, Royal Society University Research Fellow, Institute of Astronomy, University of Cambridge. \textbf{Email:} anastasia.fialkov@gmail.com \\


\end{document}
